\documentclass[tcc,capa]{texufpel}

\usepackage[utf8]{inputenc} % acentuacao
\usepackage{graphicx} % para inserir figuras
\usepackage[T1]{fontenc}

\hypersetup{
    hidelinks, % Remove coloração e caixas
    unicode=true,   %Permite acentuação no bookmark
    linktoc=all %Habilita link no nome e página do sumário
}

\unidade{Centro de Desenvolvimento Tecnológico}
\curso{Ciência da Computação}
\nomecurso{Bacharelado em Ciência da Computação}
\titulocurso{Bacharel em Ciência da Computação}

\unidadeeng{Technology Development Center}
\cursoeng{Computer Science}


\title{VideoLearnAI: LLM Powered Web Application para aprendizagem ativa com vídeos do Youtube}

\author{Aguiar}{Marilton Sanchotene de}
\advisor[Prof.~Dr.]{Aguiar}{Marilton Sanchotene de}
\coadvisor[Prof.~Dr.]{Aguiar}{Marilton Sanchotene de}
\collaborator[Prof.~Dr.]{Aguiar}{Marilton Sanchotene de}

%Palavras-chave em PT_BR
%use letras minúsculas seguidas por ;
%a última terminada por .
\keyword{palavrachave-um;}
\keyword{palavrachave-dois;}
\keyword{palavrachave-tres;}
\keyword{palavrachave-quatro.}

%Palavras-chave em EN_US
\keywordeng{keyword-one;}
\keywordeng{keyword-two;}
\keywordeng{keyword-three;}
\keywordeng{keyword-four.}

\begin{document}

%\renewcommand{\advisorname}{Orientadora}           %descomente caso tenhas orientadora
%\renewcommand{\coadvisorname}{Coorientadora}      %descomente caso tenhas coorientadora

\maketitle 

\sloppy

\fichacatalografica

%\folhadeaprovacao

%Composição da Banca Examinadora
\begin{aprovacao}{30 de fevereiro de 2019} %data da banca por extenso
\noindent Prof. Dr. Marilton Sanchotene de Aguiar (orientador)\\
Doutor em Computação pela Universidade Federal do Rio Grande do Sul.\\[1cm]

\noindent Prof. Dr. Paulo Roberto Ferreira Jr.\\
Doutor em Computação pela Universidade Federal do Rio Grande do Sul.\\[1cm]

\noindent Prof. Dr. Ricardo Matsumura Araujo\\
Doutor em Computação pela Universidade Federal do Rio Grande do Sul.\\[1cm]

\noindent Prof. Dr. Luciano da Silva Pinto\\
Doutor em Biotecnologia pela Universidade Federal de Pelotas.
\end{aprovacao}

%Opcional
\begin{dedicatoria}
  Dedico\ldots 
\end{dedicatoria}

%Opcional
\begin{agradecimentos}
  Agradeço\ldots 
\end{agradecimentos}

%Opcional
\begin{epigrafe}
  Só sei que nada sei.\\
  {\sc --- Sócrates}
\end{epigrafe}

%Resumo em Portugues (no maximo 500 palavras)
\begin{abstract}
Bla blabla blablabla bla.  Bla blabla blablabla bla.  Bla blabla
blablabla bla.  Bla blabla blablabla bla.  Bla blabla blablabla bla.
Bla blabla blablabla bla.  Bla blabla blablabla bla.  Bla blabla
blablabla bla.  Bla blabla blablabla bla.  Bla blabla blablabla bla.
Bla blabla blablabla bla.  Bla blabla blablabla bla.  Bla blabla
blablabla bla.  Bla blabla blablabla bla.  Bla blabla blablabla bla.
Bla blabla blablabla bla.  Bla blabla blablabla bla.  Bla blabla
blablabla bla.  Bla blabla blablabla bla.  Bla blabla blablabla bla.
Bla blabla blablabla bla.
\end{abstract}

%Resumo em Inglês (no maximo 500 palavras)
\begin{englishabstract}{Titulo do Trabalho em Ingles}
Bla blabla blablabla bla.  Bla blabla blablabla bla.  Bla blabla
blablabla bla.  Bla blabla blablabla bla.  Bla blabla blablabla bla.
Bla blabla blablabla bla.  Bla blabla blablabla bla.  Bla blabla
blablabla bla.  Bla blabla blablabla bla.  Bla blabla blablabla bla.
Bla blabla blablabla bla.  Bla blabla blablabla bla.  Bla blabla
blablabla bla.  Bla blabla blablabla bla.  Bla blabla blablabla bla.
Bla blabla blablabla bla.  Bla blabla blablabla bla.  Bla blabla
blablabla bla.  Bla blabla blablabla bla.  Bla blabla blablabla bla.
Bla blabla blablabla bla.
\end{englishabstract}

%Lista de Figuras
\listoffigures

%Lista de Tabelas
\listoftables

%lista de abreviaturas e siglas
\begin{listofabbrv}{ABNT}%coloque aqui a maior sigla para ajustar a distância
        \item[ABNT] Associação Brasileira de Normas Técnicas
        \item[NUMA] Non-Uniform Memory Access
        \item[SIMD] Single Instruction Multiple Data
        \item[SMP] Symmetric Multi-Processor
        \item[SPMD] Single Program Multiple Data
\end{listofabbrv}

%Sumario
\tableofcontents

\chapter{Introduction}
\section{Problem Context}
The evolution of educational technologies has radically transformed traditional learning models, shifting from passive consumption to more engaging, interactive approaches. Active learning, a methodology that encourages students to actively participate in their learning process, has gained prominence due to its proven effectiveness. Unlike traditional learning environments, where students merely receive information, active learning involves tasks such as problem-solving, discussions, and peer teaching—methods that significantly enhance understanding and retention.

However, despite the rapid advancement of digital platforms, many high-quality educational resources from leading institutions like MIT, Harvard, and Stanford remain vastly underutilized. College education is notoriously expensive, and many students are unable to afford tuition fees for top-tier institutions. Thankfully, the internet has democratized access to knowledge, offering free educational resources from prestigious colleges that would otherwise be out of reach. Platforms like YouTube host a wealth of content, including recorded lectures, research papers, and open courses.

Yet, much of this content remains passive—primarily consisting of long lecture videos or static reading materials—which limits student interaction and engagement. Without the support of a well-designed platform to help users interact with the content and actively engage in their learning, these valuable resources fail to reach their full potential. The gap between the vast amount of available content and the active learning opportunities it can provide is significant and must be addressed to truly harness the power of digital education.

\section{Justification}
Active learning is a well-established approach to improving student outcomes. According to Bonwell and Eison (1991), "Active learning is not a new fad; it is a proven strategy that enhances student engagement and helps students learn more effectively." This method has been linked to better knowledge retention, improved critical thinking, and increased motivation among learners. As educational content continues to shift to digital platforms, it becomes essential to transform passive resources into more dynamic, interactive experiences that encourage active learning.

In particular, video-based content, which has become ubiquitous in online education, offers a significant opportunity to incorporate active learning strategies. By integrating tools such as interactive quizzes, chapter summaries, and discussions into video lectures, we can create a more engaging learning environment that prompts students to actively process and apply the material. The availability of high-quality content from top universities and online platforms such as Coursera and edX presents an ideal opportunity to experiment with these approaches and unlock the full potential of digital learning.

\section{Objectives}
The primary objective of this work is to develop a platform that facilitates active learning by incorporating interactive elements into educational video content. Specifically, the objectives are:

\begin{itemize}
    \item To develop a platform that enhances video-based learning by integrating features such as chapter summaries, interactive Q\&A sessions, and true/false questions for each video segment.
    \item To utilize high-quality educational content from top universities and open-access platforms, adapting it to support active learning methods.
    \item To evaluate the effectiveness of active learning techniques on student engagement and retention through user feedback and performance testing.
    \item To explore scalability by extending the platform to a variety of subjects and content types, from technical courses to humanities.
\end{itemize}

By achieving these goals, we aim to contribute to the growing body of research on the integration of active learning with digital education platforms, helping to bridge the gap between the vast amount of available content and the need for engaging, effective learning experiences.

\section{Structure of the Work}
This document is organized as follows: Chapter 2 provides a literature review on active learning, language models, and SaaS development. Chapter 3 details the methodology, including the development strategy and tools used. Chapter 4 discusses the implementation of the platform and its features. Chapter 5 presents the results of the platform's performance and user feedback. Finally, Chapter 6 concludes the work with a discussion of the objectives achieved, lessons learned, and future improvements.


\chapter{Literature Review}
\section{Active Learning and Its Potential}
\section{Artificial Intelligence and Language Models (LLMs)}
\section{Technologies for SaaS Development}
\section{Applications of LLMs in Educational Platforms}
\section{Existing Solutions in the Market}

\chapter{Methodology}
\section{Development Strategy}
\section{Tools and Technologies Used}
\section{Full Stack Application Architecture}
\section{Success Criteria and Evaluation}

\chapter{Application Development}
\section{Planning and Definition of Features}
\section{Front-End Structure}
\section{Back-End Structure and Integrations}
\section{Implementation of LLMs in the Application}
\section{Implementation of Interactive Features}
\section{Challenges Faced and Solutions Adopted}

\chapter{Results}
\section{Platform Functionality}
\section{Performance and Scalability}
\section{User Feedback and Testing}
\section{Practical Demonstration}

\chapter{Conclusion}
\section{Objectives Achieved}
\section{Lessons Learned}
\section{Future Improvements and Scalability Plans}
\section{Final Considerations}

\chapter{References}

\chapter{Appendices}
\section{Source Code}
\section{Technical Documentation}




% Bibliografia http://liinwww.ira.uka.de/bibliography/index.html um
% site que cataloga no formato bibtex a bibliografia em computacao
% \bibliography{nomedoarquivo.bib} (sem extensao)
% \bibliographystyle{formato.bst} (sem extensao)

\bibliographystyle{abnt}
\bibliography{bibliografia} 

% Apêndices (Opcional) - Material produzido pelo autor
\apendices
\chapter{Um Apêndice}

% Anexos (Opcional) - Material produzido por outro
\anexos
\chapter{Um Anexo}

Bla blabla blablabla bla.  Bla blabla blablabla bla.  Bla blabla
blablabla bla.  Bla blabla blablabla bla.  Bla blabla blablabla bla.
Bla blabla blablabla bla.  Bla blabla blablabla bla.  Bla blabla
blablabla bla.  Bla blabla blablabla bla.  Bla blabla blablabla bla.
Bla blabla blablabla bla.  Bla blabla blablabla bla.  Bla blabla
blablabla bla.  Bla blabla blablabla bla.  Bla blabla blablabla bla.
Bla blabla blablabla bla.  Bla blabla blablabla bla.  Bla blabla
blablabla bla.  Bla blabla blablabla bla.  Bla blabla blablabla bla.
Bla blabla blablabla bla.

Bla blabla blablabla bla.  Bla blabla blablabla bla.  Bla blabla
blablabla bla.  Bla blabla blablabla bla.  Bla blabla blablabla bla.
Bla blabla blablabla bla.  Bla blabla blablabla bla.  Bla blabla
blablabla bla.  Bla blabla blablabla bla.  Bla blabla blablabla bla.
Bla blabla blablabla bla.  Bla blabla blablabla bla.  Bla blabla
blablabla bla.  Bla blabla blablabla bla.  Bla blabla blablabla bla.
Bla blabla blablabla bla.  Bla blabla blablabla bla.  Bla blabla
blablabla bla.  Bla blabla blablabla bla.  Bla blabla blablabla bla.
Bla blabla blablabla bla.

\chapter{Outro Anexo}

Bla blabla blablabla bla.  Bla blabla blablabla bla.  Bla blabla
blablabla bla.  Bla blabla blablabla bla.  Bla blabla blablabla bla.
Bla blabla blablabla bla.  Bla blabla blablabla bla.  Bla blabla
blablabla bla.  Bla blabla blablabla bla.  Bla blabla blablabla bla.
Bla blabla blablabla bla.  Bla blabla blablabla bla.  Bla blabla
blablabla bla.  Bla blabla blablabla bla.  Bla blabla blablabla bla.
Bla blabla blablabla bla.  Bla blabla blablabla bla.  Bla blabla
blablabla bla.  Bla blabla blablabla bla.  Bla blabla blablabla bla.
Bla blabla blablabla bla.

Bla blabla blablabla bla.  Bla blabla blablabla bla.  Bla blabla
blablabla bla.  Bla blabla blablabla bla.  Bla blabla blablabla bla.
Bla blabla blablabla bla.  Bla blabla blablabla bla.  Bla blabla
blablabla bla.  Bla blabla blablabla bla.  Bla blabla blablabla bla.
Bla blabla blablabla bla.  Bla blabla blablabla bla.  Bla blabla
blablabla bla.  Bla blabla blablabla bla.  Bla blabla blablabla bla.
Bla blabla blablabla bla.  Bla blabla blablabla bla.  Bla blabla
blablabla bla.  Bla blabla blablabla bla.  Bla blabla blablabla bla.
Bla blabla blablabla bla.




\end{document}

