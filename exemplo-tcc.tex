\documentclass[tcc,capa]{texufpel}

\usepackage[utf8]{inputenc} % acentuacao
\usepackage{graphicx} % para inserir figuras
\usepackage[T1]{fontenc}

\hypersetup{
    hidelinks, % Remove coloração e caixas
    unicode=true,   %Permite acentuação no bookmark
    linktoc=all %Habilita link no nome e página do sumário
}

\unidade{Centro de Desenvolvimento Tecnológico}
\curso{Ciência da Computação}
\nomecurso{Bacharelado em Ciência da Computação}
\titulocurso{Bacharel em Ciência da Computação}

\unidadeeng{Technology Development Center}
\cursoeng{Computer Science}


\title{LLM-Powered Applications: Tecnologia, Questões e estudo de caso.}

\author{Aguiar}{Marilton Sanchotene de}
\advisor[Prof.~Dr.]{Aguiar}{Marilton Sanchotene de}
\coadvisor[Prof.~Dr.]{Aguiar}{Marilton Sanchotene de}
\collaborator[Prof.~Dr.]{Aguiar}{Marilton Sanchotene de}

%Palavras-chave em PT_BR
%use letras minúsculas seguidas por ;
%a última terminada por .
\keyword{palavrachave-um;}
\keyword{palavrachave-dois;}
\keyword{palavrachave-tres;}
\keyword{palavrachave-quatro.}

%Palavras-chave em EN_US
\keywordeng{keyword-one;}
\keywordeng{keyword-two;}
\keywordeng{keyword-three;}
\keywordeng{keyword-four.}

\begin{document}

%\renewcommand{\advisorname}{Orientadora}           %descomente caso tenhas orientadora
%\renewcommand{\coadvisorname}{Coorientadora}      %descomente caso tenhas coorientadora

\maketitle 

\sloppy

\fichacatalografica

%\folhadeaprovacao

%Composição da Banca Examinadora
\begin{aprovacao}{30 de fevereiro de 2019} %data da banca por extenso
\noindent Prof. Dr. Marilton Sanchotene de Aguiar (orientador)\\
Doutor em Computação pela Universidade Federal do Rio Grande do Sul.\\[1cm]

\noindent Prof. Dr. Paulo Roberto Ferreira Jr.\\
Doutor em Computação pela Universidade Federal do Rio Grande do Sul.\\[1cm]

\noindent Prof. Dr. Ricardo Matsumura Araujo\\
Doutor em Computação pela Universidade Federal do Rio Grande do Sul.\\[1cm]

\noindent Prof. Dr. Luciano da Silva Pinto\\
Doutor em Biotecnologia pela Universidade Federal de Pelotas.
\end{aprovacao}

%Opcional
\begin{dedicatoria}
  Dedico\ldots 
\end{dedicatoria}

%Opcional
\begin{agradecimentos}
  Agradeço\ldots 
\end{agradecimentos}

%Opcional
\begin{epigrafe}
  Só sei que nada sei.\\
  {\sc --- Sócrates}
\end{epigrafe}

%Resumo em Portugues (no maximo 500 palavras)
\begin{abstract}
Bla blabla blablabla bla.  Bla blabla blablabla bla.  Bla blabla
blablabla bla.  Bla blabla blablabla bla.  Bla blabla blablabla bla.
Bla blabla blablabla bla.  Bla blabla blablabla bla.  Bla blabla
blablabla bla.  Bla blabla blablabla bla.  Bla blabla blablabla bla.
Bla blabla blablabla bla.  Bla blabla blablabla bla.  Bla blabla
blablabla bla.  Bla blabla blablabla bla.  Bla blabla blablabla bla.
Bla blabla blablabla bla.  Bla blabla blablabla bla.  Bla blabla
blablabla bla.  Bla blabla blablabla bla.  Bla blabla blablabla bla.
Bla blabla blablabla bla.
\end{abstract}

%Resumo em Inglês (no maximo 500 palavras)
\begin{englishabstract}{Titulo do Trabalho em Ingles}
Bla blabla blablabla bla.  Bla blabla blablabla bla.  Bla blabla
blablabla bla.  Bla blabla blablabla bla.  Bla blabla blablabla bla.
Bla blabla blablabla bla.  Bla blabla blablabla bla.  Bla blabla
blablabla bla.  Bla blabla blablabla bla.  Bla blabla blablabla bla.
Bla blabla blablabla bla.  Bla blabla blablabla bla.  Bla blabla
blablabla bla.  Bla blabla blablabla bla.  Bla blabla blablabla bla.
Bla blabla blablabla bla.  Bla blabla blablabla bla.  Bla blabla
blablabla bla.  Bla blabla blablabla bla.  Bla blabla blablabla bla.
Bla blabla blablabla bla.
\end{englishabstract}

%Lista de Figuras
\listoffigures

%Lista de Tabelas
\listoftables

%lista de abreviaturas e siglas
\begin{listofabbrv}{ABNT}%coloque aqui a maior sigla para ajustar a distância
        \item[ABNT] Associação Brasileira de Normas Técnicas
        \item[NUMA] Non-Uniform Memory Access
        \item[SIMD] Single Instruction Multiple Data
        \item[SMP] Symmetric Multi-Processor
        \item[SPMD] Single Program Multiple Data
\end{listofabbrv}

%Sumario
\tableofcontents

\chapter{Introdução}

\section{Grandes Modelos de Linguagem}

Bla blabla blablabla bla.  Bla blabla blablabla bla.  Bla blabla
blablabla bla.  Bla blabla blablabla bla.  Bla blabla blablabla bla.
Bla blabla blablabla bla.  Bla blabla blablabla bla.  Bla blabla
blablabla bla.  Bla blabla blablabla bla.  Bla blabla blablabla bla.
Bla blabla blablabla bla.

\begin{quote}
  Bla blabla blablabla bla.  Bla blabla blablabla bla.  Bla blabla
  blablabla bla.  Bla blabla blablabla bla.  Bla blabla blablabla bla.
  Bla blabla blablabla bla.  Bla blabla blablabla bla.  Bla blabla
  blablabla bla.  Bla blabla blablabla bla.  Bla blabla blablabla
  bla~\citet{Moore:1979:MAI,Aguiar:2005}.
\end{quote}
  
Bla blabla blablabla bla.  Bla blabla blablabla bla.  Bla blabla
blablabla bla.  Bla blabla blablabla bla.  Bla blabla blablabla bla.
Bla blabla blablabla bla.  Bla blabla blablabla bla.  Bla blabla
blablabla bla.  Bla blabla blablabla bla.  Bla blabla blablabla bla.
Bla blabla blablabla bla.  Bla blabla blablabla bla.  Bla blabla
blablabla bla.  Bla blabla blablabla bla.  Bla blabla blablabla bla.
Bla blabla blablabla bla.  Bla blabla blablabla bla.  Bla blabla
blablabla bla.  Bla blabla blablabla bla.  Bla blabla blablabla bla.
Bla blabla blablabla bla.

Bla blabla blablabla bla.  Bla blabla blablabla bla.  Bla blabla
blablabla bla.  Bla blabla blablabla bla.  Bla blabla blablabla bla.
Bla blabla blablabla bla.  Bla blabla blablabla bla.  Bla blabla
blablabla bla.  Bla blabla blablabla bla.  Bla blabla blablabla bla.
Bla blabla blablabla bla.  Bla blabla blablabla bla.  Bla blabla
blablabla bla.  Bla blabla blablabla bla.  Bla blabla blablabla bla.
Bla blabla blablabla bla.  Bla blabla blablabla bla.  Bla blabla
blablabla bla.  Bla blabla blablabla bla.  Bla blabla blablabla bla.
Bla blabla blablabla bla~\cite{vonNeumann:1966:TSR}.

\section{Outra seção}

Bla blabla blablabla bla.  Bla blabla blablabla bla.  Bla blabla
blablabla bla.  Bla blabla blablabla bla.  Bla blabla blablabla bla.
Bla blabla blablabla bla.  Bla blabla blablabla bla.  Bla blabla
blablabla bla.  Bla blabla blablabla bla.  Bla blabla blablabla bla.
Bla blabla blablabla bla.  Bla blabla blablabla bla.  Bla blabla
blablabla bla.  Bla blabla blablabla bla.  Bla blabla blablabla bla.
Bla blabla blablabla bla.  Bla blabla blablabla bla.  Bla blabla
blablabla bla.  Bla blabla blablabla bla.  Bla blabla blablabla bla.
Bla blabla blablabla bla~\ref{tabela}.

\begin{table}
\caption{Nome da Tabela}\label{tabela2}
\centering\begin{tabular}{p{4cm}p{5cm}p{6cm}}
\hline
Blabla & Blabla & Blablabla\\
\hline
{\small Bla} & {\small Blabla} & {\small\em Bla blabla blablabla blabla
  blablabla blabla blablabla.}\\
{\small Bla} & {\small Blabla} & {\small\em Bla blabla blablabla blabla
  blablabla blabla blablabla.}\\
{\small Bla} & {\small Blabla} & {\small\em Bla blabla blablabla blabla
  blablabla blabla blablabla.}\\
{\small Bla} & {\small Blabla} & {\small\em Bla blabla blablabla blabla
  blablabla blabla blablabla.}\\
{\small Bla} & {\small Blabla} & {\small\em Bla blabla blablabla blabla
  blablabla blabla blablabla.}\\
{\small Bla} & {\small Blabla} & {\small\em Bla blabla blablabla blabla
  blablabla blabla blablabla. Conforme a figura~\ref{figura}}\\
\hline
\end{tabular}
\begin{flushleft}
{\small Fonte: Elaborada pelo autor.}    
\end{flushleft}
\end{table}


\subsection{Uma subseção}

Bla blabla blablabla bla.  Bla blabla blablabla bla.  Bla blabla
blablabla bla.  Bla blabla blablabla bla.  Bla blabla blablabla bla.
Bla blabla blablabla bla.  Bla blabla blablabla bla.  Bla blabla
blablabla bla.  Bla blabla blablabla bla.  Bla blabla blablabla bla.
Bla blabla blablabla bla.  Bla blabla blablabla bla.  Bla blabla
blablabla bla.  Bla blabla blablabla bla.  Bla blabla blablabla bla.
Bla blabla blablabla bla.  Bla blabla blablabla bla.  Bla blabla
blablabla bla.  Bla blabla blablabla bla.  Bla blabla blablabla bla.
Bla blabla blablabla bla.

\chapter{LLM Powered applications}

Bla blabla blablabla bla.  Bla blabla blablabla bla.  Bla blabla
blablabla bla.  Bla blabla blablabla bla.  Bla blabla blablabla bla.
Bla blabla blablabla bla.  Bla blabla blablabla bla.  Bla blabla
blablabla bla.  Bla blabla blablabla bla.  Bla blabla blablabla bla.
Bla blabla blablabla bla.  Bla blabla blablabla bla.  Bla blabla
blablabla bla.  Bla blabla blablabla bla.  Bla blabla blablabla bla.
Bla blabla blablabla bla.  Bla blabla blablabla bla.  Bla blabla
blablabla bla.  Bla blabla blablabla bla.  Bla blabla blablabla bla.
Bla blabla blablabla bla~\ref{tabela2}.

\begin{table}
  \begin{center}
    \caption{Nome da Tabela}\label{tabela2}
    \begin{tabular}{p{4cm}p{5cm}p{6cm}}
      \hline
      Blabla & Blabla & Blablabla\\
      \hline
      {\small Bla} & {\small Blabla} & {\small\em Bla blabla blablabla blabla
                                       blablabla blabla blablabla.}\\
      {\small Bla} & {\small Blabla} & {\small\em Bla blabla blablabla blabla
                                       blablabla blabla blablabla.}\\
      {\small Bla} & {\small Blabla} & {\small\em Bla blabla blablabla blabla
                                       blablabla blabla blablabla.}\\
      {\small Bla} & {\small Blabla} & {\small\em Bla blabla blablabla blabla
                                       blablabla blabla blablabla.}\\
      {\small Bla} & {\small Blabla} & {\small\em Bla blabla blablabla blabla
                                       blablabla blabla blablabla.}\\
      {\small Bla} & {\small Blabla} & {\small\em Bla blabla blablabla blabla
                                       blablabla blabla blablabla. Conforme a figura~\ref{figura}}\\
      \hline
    \end{tabular}
  \end{center}
\end{table}

\begin{figure}[htbp]
\caption{Nome da figura}   
\centering \includegraphics[scale=.4]{figura}\\
\begin{flushleft}
{\small Fonte: Elaborada pelo autor.}    
\end{flushleft}
\label{figura}
\end{figure}


\section{Fine tuning}
\section{Retrieval Augmented Generation}
\section{Engenharia de prompt}
A engenharia de prompt é uma disciplina emergente que desempenha um papel fundamental na utilização e eficiência de modelos de linguagem grandes (LLMs), especialmente dentro do domínio das tarefas de processamento de linguagem natural. A medida que a IA continua a permear diversos setores, a habilidade de comunicar efetivamente com esses sistemas torna-se crucial. Este capítulo tem como objetivo dissecar o conceito de engenharia de prompt, suas técnicas, aplicações e o profundo impacto que ela tem em diferentes campos.

\subsection*{Técnicas de Engenharia de Prompt}
Essencialmente, a engenharia de prompt é a prática de formular e refinar prompts para maximizar o desempenho de modelos de linguagem. Um prompt bem elaborado pode alterar significativamente a saída de uma IA, fazendo a diferença entre uma resposta útil e uma irrelevante.

A engenharia de prompt eficaz depende de vários princípios centrais que garantem clareza, especificidade e relevância na interação com sistemas de IA. Esses princípios incluem escrever instruções claras e descritivas, usar delimitadores, fornecer exemplos, atribuir papéis, adicionar informações de contexto, dividir tarefas complexas e solicitar múltiplas soluções. Técnicas avançadas também envolvem a sinergia de raciocínio e ação em modelos de linguagem, como modelos auxiliados por programas, raciocínio automático e uso de ferramentas, e ajuste de prompt para adaptar as respostas do modelo a necessidades específicas (\href{https://daveai.substack.com/p/prompt-engineering-full-guide}{DaveAI}).

\subsubsection*{Técnicas Avançadas de Prompt}

As técnicas avançadas de prompt englobam uma variedade de estratégias projetadas para obter respostas mais sofisticadas de LLMs. Estas incluem prompt de encadeamento de pensamento, que encoraja os modelos a exibir seu processo de raciocínio, e prompt de menor para maior, que guia os modelos por meio de uma série progressiva de complexidade em suas respostas. Além disso, o prompt de geração de conhecimento e o uso de modelos de árvore de pensamento representam abordagens inovadoras para a resolução de problemas com IA (\href{https://arxiv.org/abs/2210.03629}{Arxiv}).

\subsection*{Aplicações da Engenharia de Prompt}

As aplicações da engenharia de prompt são vastas e variadas. Na escrita acadêmica, os pesquisadores podem aproveitar a engenharia de prompt para simplificar revisões de literatura, sintetizar informações complexas e até mesmo gerar rascunhos de artigos. Além da academia, a engenharia de prompt é instrumental em áreas como atendimento ao cliente, onde a IA pode fornecer respostas personalizadas a perguntas, e em indústrias criativas, onde a IA pode auxiliar na criação de conteúdo que varia da escrita à geração de imagens (\href{https://intellyverse.com/blog/prompt-engineering}{Intellyverse}).

\subsection*{Impacto da Engenharia de Prompt}

O impacto da engenharia de prompt é substancial, pois influencia diretamente a eficácia das interações de IA. Ao otimizar prompts, os usuários podem obter saídas mais precisas e contextualmente relevantes dos sistemas de IA, aumentando assim a produtividade e reduzindo o potencial de mal-entendidos. O campo também levanta importantes considerações sobre o uso ético da IA, já que os prompts em si podem moldar a natureza das informações fornecidas pela IA (\href{https://link.springer.com/content/pdf/10.1007/s10439-023-03272-4.pdf?pdf=button}{Giray}).

\subsection*{Conclusão}

A engenharia de prompt não é apenas uma habilidade técnica, mas uma forma de arte que requer um profundo entendimento tanto da linguagem quanto da tecnologia. À medida que a IA se torna cada vez mais sofisticada, o papel do engenheiro de prompt se tornará ainda mais crítico na moldagem das interações entre humanos e máquinas. A disciplina está no cruzamento da comunicação, tecnologia e criatividade, oferecendo uma nova fronteira para exploração e inovação na era da IA.

\section*{Lista de Referências}

\begin{itemize}
  \item Giray, Louie. "Engenharia de Prompt com ChatGPT: Um Guia para Escritores Acadêmicos." \textit{Springer Nature}, 2023, \url{https://link.springer.com/article/10.1007/s10439-023-03272-4}.
  \item "Como se Tornar um Engenheiro de Prompt." \textit{DataCamp}, \url{https://www.datacamp.com/blog/how-to-become-a-prompt-engineer}.
  \item "Guia Completo de Engenharia de Prompt." \textit{DaveAI}, \url{https://daveai.substack.com/p/prompt-engineering-full-guide}.
  \item "Sinergizando Raciocínio e Ação em Modelos de Linguagem." \textit{Arxiv}, \url{https://arxiv.org/abs/2210.03629}.
  \item "Engenharia de Prompt." \textit{Intellyverse}, \url{https://intellyverse.com/blog/prompt-engineering}.
\end{itemize}






\section{Function Calling e Output Parsers}
\section{Agentes}
\chapter{Desenvolvimento de uma LLM Powered Application para facilitar o consumo e aprendizagem através de vídeos}

\section{Motivação}

A aprendizagem ativa tem sido objeto de vários artigos acadêmicos importantes, destacando seu impacto positivo nos resultados dos alunos. A pesquisa de Freeman et al. (2014) demonstrou que a aprendizagem ativa pode aumentar significativamente as notas dos alunos em relação aos métodos didáticos, com alunos em cursos sem aprendizagem ativa sendo 1,5 vezes mais propensos a reprovar do que aqueles com aprendizagem ativa. Além disso, a aprendizagem ativa tem sido mostrada positiva para melhorar a motivação dos alunos, habilidades de pensamento crítico, retenção de informações e habilidades interpessoais. Por outro lado, a aprendizagem passiva, como assistir a vídeos, tem sido associada a desvantagens, como níveis mais baixos de engajamento e compreensão superficial de conceitos-chave.

Além disso, dado o contexto de excesso de dados e nossa crescente dificuldade em navegar em meio a um mar de informações, cada vez mais se faz necessário tecnologias que possam ajudar com esses problemas contemporâneos.

É importante ressaltar também que muitas vezes o conteúdo de interesse está disponível em outra língua, dificultando a aprendizagem. Apesar do Youtube possibilitar a geração de legendas automáticas, a legenda gerada por esse método não possui tanta qualidade quanto por exemplo com os modelos mais avançados de Reconhecimento de Voz Automático (ASR).

A criação de uma aplicação potencializada por LLMs e ASR, como o Whisper, tem o poder de transformar vídeos educacionais em experiências de aprendizagem mais ricas e interativas. Ao incorporar funcionalidades como a geração automática de capítulos, facilitação de perguntas e respostas por capítulos, e o uso de Retrieval Augmented Generation (RAG) para melhoria do QA em vídeos, esta aplicação não apenas melhora a acessibilidade e a personalização do conteúdo, mas também promove a aprendizagem ativa e engajada. Ademais, a transcrição e geração automática de legendas tornam o conteúdo acessível para um público mais amplo, independentemente do idioma nativo do espectador, ampliando assim o alcance e a eficácia da educação por meio de vídeos.




\section{Features}
\subsection{Improved Readability}
\subsection{Transcrição/Tradução com Whisper}
\subsection{Auto Chapter}
\subsection{Geração de perguntas por capítulo}



\chapter{Conclusão}



% Bibliografia http://liinwww.ira.uka.de/bibliography/index.html um
% site que cataloga no formato bibtex a bibliografia em computacao
% \bibliography{nomedoarquivo.bib} (sem extensao)
% \bibliographystyle{formato.bst} (sem extensao)

\bibliographystyle{abnt}
\bibliography{bibliografia} 

% Apêndices (Opcional) - Material produzido pelo autor
\apendices
\chapter{Um Apêndice}

% Anexos (Opcional) - Material produzido por outro
\anexos
\chapter{Um Anexo}

Bla blabla blablabla bla.  Bla blabla blablabla bla.  Bla blabla
blablabla bla.  Bla blabla blablabla bla.  Bla blabla blablabla bla.
Bla blabla blablabla bla.  Bla blabla blablabla bla.  Bla blabla
blablabla bla.  Bla blabla blablabla bla.  Bla blabla blablabla bla.
Bla blabla blablabla bla.  Bla blabla blablabla bla.  Bla blabla
blablabla bla.  Bla blabla blablabla bla.  Bla blabla blablabla bla.
Bla blabla blablabla bla.  Bla blabla blablabla bla.  Bla blabla
blablabla bla.  Bla blabla blablabla bla.  Bla blabla blablabla bla.
Bla blabla blablabla bla.

Bla blabla blablabla bla.  Bla blabla blablabla bla.  Bla blabla
blablabla bla.  Bla blabla blablabla bla.  Bla blabla blablabla bla.
Bla blabla blablabla bla.  Bla blabla blablabla bla.  Bla blabla
blablabla bla.  Bla blabla blablabla bla.  Bla blabla blablabla bla.
Bla blabla blablabla bla.  Bla blabla blablabla bla.  Bla blabla
blablabla bla.  Bla blabla blablabla bla.  Bla blabla blablabla bla.
Bla blabla blablabla bla.  Bla blabla blablabla bla.  Bla blabla
blablabla bla.  Bla blabla blablabla bla.  Bla blabla blablabla bla.
Bla blabla blablabla bla.

\chapter{Outro Anexo}

Bla blabla blablabla bla.  Bla blabla blablabla bla.  Bla blabla
blablabla bla.  Bla blabla blablabla bla.  Bla blabla blablabla bla.
Bla blabla blablabla bla.  Bla blabla blablabla bla.  Bla blabla
blablabla bla.  Bla blabla blablabla bla.  Bla blabla blablabla bla.
Bla blabla blablabla bla.  Bla blabla blablabla bla.  Bla blabla
blablabla bla.  Bla blabla blablabla bla.  Bla blabla blablabla bla.
Bla blabla blablabla bla.  Bla blabla blablabla bla.  Bla blabla
blablabla bla.  Bla blabla blablabla bla.  Bla blabla blablabla bla.
Bla blabla blablabla bla.

Bla blabla blablabla bla.  Bla blabla blablabla bla.  Bla blabla
blablabla bla.  Bla blabla blablabla bla.  Bla blabla blablabla bla.
Bla blabla blablabla bla.  Bla blabla blablabla bla.  Bla blabla
blablabla bla.  Bla blabla blablabla bla.  Bla blabla blablabla bla.
Bla blabla blablabla bla.  Bla blabla blablabla bla.  Bla blabla
blablabla bla.  Bla blabla blablabla bla.  Bla blabla blablabla bla.
Bla blabla blablabla bla.  Bla blabla blablabla bla.  Bla blabla
blablabla bla.  Bla blabla blablabla bla.  Bla blabla blablabla bla.
Bla blabla blablabla bla.




\end{document}

